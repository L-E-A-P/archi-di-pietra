%!TEX TS-program = xelatex
%!TEX encoding = UTF-8 Unicode

\documentclass[tikz]{standalone}

\usetikzlibrary{decorations.pathreplacing,calligraphy}

\usepackage{iwona}
\usepackage[T1]{fontenc}

\linespread{1}

\usepackage{fontspec,
			xltxtra,
			xunicode,
			unicode-math
			}

\setromanfont[
	Mapping=tex-text
]{Datalegreya}

% PANTONE SOLID UNCOATED
\definecolor{PinkU}{RGB}{209,66,141}
\definecolor{YellowU}{RGB}{255,232,0}
\definecolor{3165U}{RGB}{47,92,102}
\definecolor{d0a399}{HTML}{d0a399}

%-------------------------------------------------------------------------------
%------------------------------------------------------------ BEGIN DOCUMENT ---
%-------------------------------------------------------------------------------
\begin{document}

\begin{tikzpicture}[scale=1.1]  
    \draw[fill=PinkU] (0,0) rectangle (30,30);

	\node[
	  anchor=west,
	  baseline,
	  align=left,
	  font=\fontsize{100}{70}\selectfont
	] at (1.5,25) [color=YellowU,fill=none] {

// ARCHI DI PIETRA\\
// secondo respiro\\
};    
    
	\node[
	  anchor=west,
	  baseline,
	  align=left,
	  font=\fontsize{51}{40}\selectfont
	] at (1.5,17) [color=YellowU,fill=none] {

lontano = 22 + 1 + 2024 : @(16)\\
~~with (\\
~~~~francesco = galante;\\
);\\[1cm]
process = lontano @(leap);
};

	\node[
	  anchor=west,
	  baseline,
	  align=left,
	  font=\fontsize{51}{40}\selectfont
	] at (1.5,4) [color=YellowU,fill=none] {

// ARCHI DI PIETRA\\
// APPRODI ALLA LETTERATURA MUSICALE\\
// LEAP\\
// VIA LUIGI GAETANO MARINI 12 ROMA
};            
\end{tikzpicture}

\end{document}
